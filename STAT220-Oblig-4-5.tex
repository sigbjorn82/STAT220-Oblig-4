\PassOptionsToPackage{unicode=true}{hyperref} % options for packages loaded elsewhere
\PassOptionsToPackage{hyphens}{url}
%
\documentclass[]{article}
\usepackage{lmodern}
\usepackage{amssymb,amsmath}
\usepackage{ifxetex,ifluatex}
\usepackage{fixltx2e} % provides \textsubscript
\ifnum 0\ifxetex 1\fi\ifluatex 1\fi=0 % if pdftex
  \usepackage[T1]{fontenc}
  \usepackage[utf8]{inputenc}
  \usepackage{textcomp} % provides euro and other symbols
\else % if luatex or xelatex
  \usepackage{unicode-math}
  \defaultfontfeatures{Ligatures=TeX,Scale=MatchLowercase}
\fi
% use upquote if available, for straight quotes in verbatim environments
\IfFileExists{upquote.sty}{\usepackage{upquote}}{}
% use microtype if available
\IfFileExists{microtype.sty}{%
\usepackage[]{microtype}
\UseMicrotypeSet[protrusion]{basicmath} % disable protrusion for tt fonts
}{}
\IfFileExists{parskip.sty}{%
\usepackage{parskip}
}{% else
\setlength{\parindent}{0pt}
\setlength{\parskip}{6pt plus 2pt minus 1pt}
}
\usepackage{hyperref}
\hypersetup{
            pdftitle={STAT220 Oblig 4-5},
            pdfauthor={Sigbjørn Fjelland},
            pdfborder={0 0 0},
            breaklinks=true}
\urlstyle{same}  % don't use monospace font for urls
\usepackage[margin=1in]{geometry}
\usepackage{color}
\usepackage{fancyvrb}
\newcommand{\VerbBar}{|}
\newcommand{\VERB}{\Verb[commandchars=\\\{\}]}
\DefineVerbatimEnvironment{Highlighting}{Verbatim}{commandchars=\\\{\}}
% Add ',fontsize=\small' for more characters per line
\usepackage{framed}
\definecolor{shadecolor}{RGB}{248,248,248}
\newenvironment{Shaded}{\begin{snugshade}}{\end{snugshade}}
\newcommand{\AlertTok}[1]{\textcolor[rgb]{0.94,0.16,0.16}{#1}}
\newcommand{\AnnotationTok}[1]{\textcolor[rgb]{0.56,0.35,0.01}{\textbf{\textit{#1}}}}
\newcommand{\AttributeTok}[1]{\textcolor[rgb]{0.77,0.63,0.00}{#1}}
\newcommand{\BaseNTok}[1]{\textcolor[rgb]{0.00,0.00,0.81}{#1}}
\newcommand{\BuiltInTok}[1]{#1}
\newcommand{\CharTok}[1]{\textcolor[rgb]{0.31,0.60,0.02}{#1}}
\newcommand{\CommentTok}[1]{\textcolor[rgb]{0.56,0.35,0.01}{\textit{#1}}}
\newcommand{\CommentVarTok}[1]{\textcolor[rgb]{0.56,0.35,0.01}{\textbf{\textit{#1}}}}
\newcommand{\ConstantTok}[1]{\textcolor[rgb]{0.00,0.00,0.00}{#1}}
\newcommand{\ControlFlowTok}[1]{\textcolor[rgb]{0.13,0.29,0.53}{\textbf{#1}}}
\newcommand{\DataTypeTok}[1]{\textcolor[rgb]{0.13,0.29,0.53}{#1}}
\newcommand{\DecValTok}[1]{\textcolor[rgb]{0.00,0.00,0.81}{#1}}
\newcommand{\DocumentationTok}[1]{\textcolor[rgb]{0.56,0.35,0.01}{\textbf{\textit{#1}}}}
\newcommand{\ErrorTok}[1]{\textcolor[rgb]{0.64,0.00,0.00}{\textbf{#1}}}
\newcommand{\ExtensionTok}[1]{#1}
\newcommand{\FloatTok}[1]{\textcolor[rgb]{0.00,0.00,0.81}{#1}}
\newcommand{\FunctionTok}[1]{\textcolor[rgb]{0.00,0.00,0.00}{#1}}
\newcommand{\ImportTok}[1]{#1}
\newcommand{\InformationTok}[1]{\textcolor[rgb]{0.56,0.35,0.01}{\textbf{\textit{#1}}}}
\newcommand{\KeywordTok}[1]{\textcolor[rgb]{0.13,0.29,0.53}{\textbf{#1}}}
\newcommand{\NormalTok}[1]{#1}
\newcommand{\OperatorTok}[1]{\textcolor[rgb]{0.81,0.36,0.00}{\textbf{#1}}}
\newcommand{\OtherTok}[1]{\textcolor[rgb]{0.56,0.35,0.01}{#1}}
\newcommand{\PreprocessorTok}[1]{\textcolor[rgb]{0.56,0.35,0.01}{\textit{#1}}}
\newcommand{\RegionMarkerTok}[1]{#1}
\newcommand{\SpecialCharTok}[1]{\textcolor[rgb]{0.00,0.00,0.00}{#1}}
\newcommand{\SpecialStringTok}[1]{\textcolor[rgb]{0.31,0.60,0.02}{#1}}
\newcommand{\StringTok}[1]{\textcolor[rgb]{0.31,0.60,0.02}{#1}}
\newcommand{\VariableTok}[1]{\textcolor[rgb]{0.00,0.00,0.00}{#1}}
\newcommand{\VerbatimStringTok}[1]{\textcolor[rgb]{0.31,0.60,0.02}{#1}}
\newcommand{\WarningTok}[1]{\textcolor[rgb]{0.56,0.35,0.01}{\textbf{\textit{#1}}}}
\usepackage{graphicx,grffile}
\makeatletter
\def\maxwidth{\ifdim\Gin@nat@width>\linewidth\linewidth\else\Gin@nat@width\fi}
\def\maxheight{\ifdim\Gin@nat@height>\textheight\textheight\else\Gin@nat@height\fi}
\makeatother
% Scale images if necessary, so that they will not overflow the page
% margins by default, and it is still possible to overwrite the defaults
% using explicit options in \includegraphics[width, height, ...]{}
\setkeys{Gin}{width=\maxwidth,height=\maxheight,keepaspectratio}
\setlength{\emergencystretch}{3em}  % prevent overfull lines
\providecommand{\tightlist}{%
  \setlength{\itemsep}{0pt}\setlength{\parskip}{0pt}}
\setcounter{secnumdepth}{0}
% Redefines (sub)paragraphs to behave more like sections
\ifx\paragraph\undefined\else
\let\oldparagraph\paragraph
\renewcommand{\paragraph}[1]{\oldparagraph{#1}\mbox{}}
\fi
\ifx\subparagraph\undefined\else
\let\oldsubparagraph\subparagraph
\renewcommand{\subparagraph}[1]{\oldsubparagraph{#1}\mbox{}}
\fi

% set default figure placement to htbp
\makeatletter
\def\fps@figure{htbp}
\makeatother


\title{STAT220 Oblig 4-5}
\author{Sigbjørn Fjelland}
\date{10/7/2020}

\begin{document}
\maketitle

\underline{Problem 6.1}

Vi har gitt transition matrix P:

\begin{Shaded}
\begin{Highlighting}[]
\KeywordTok{library}\NormalTok{(matrixcalc)}
\NormalTok{P=}\KeywordTok{matrix}\NormalTok{(}\KeywordTok{c}\NormalTok{(}\FloatTok{0.7}\NormalTok{, }\FloatTok{0.2}\NormalTok{, }\FloatTok{0.1}\NormalTok{, }\FloatTok{0.0}\NormalTok{, }\FloatTok{0.6}\NormalTok{, }\FloatTok{0.4}\NormalTok{, }\FloatTok{0.5}\NormalTok{, }\FloatTok{0.0}\NormalTok{, }\FloatTok{0.5}\NormalTok{), }\DataTypeTok{nrow=}\DecValTok{3}\NormalTok{, }\DataTypeTok{ncol=}\DecValTok{3}\NormalTok{, }\DataTypeTok{byrow =} \OtherTok{TRUE}\NormalTok{)}
\KeywordTok{dimnames}\NormalTok{(P)=}\KeywordTok{list}\NormalTok{(}\KeywordTok{c}\NormalTok{(}\StringTok{'0'}\NormalTok{,}\StringTok{'1'}\NormalTok{,}\StringTok{'2'}\NormalTok{), }\KeywordTok{c}\NormalTok{(}\StringTok{'0'}\NormalTok{,}\StringTok{'1'}\NormalTok{,}\StringTok{'2'}\NormalTok{))}
\KeywordTok{print}\NormalTok{(P)}
\end{Highlighting}
\end{Shaded}

\begin{verbatim}
##     0   1   2
## 0 0.7 0.2 0.1
## 1 0.0 0.6 0.4
## 2 0.5 0.0 0.5
\end{verbatim}

\begin{enumerate}
\def\labelenumi{\alph{enumi})}
\item
  Since \(P_{i,j}^{m} = P(x_{m+n}=j|x_{n})\) and \(P^{(m)}=P^{m}\):

  \(\Rightarrow{P^{(3)}=P^{3}}\) og vi får følgende matrise
\end{enumerate}

\begin{Shaded}
\begin{Highlighting}[]
\NormalTok{P_}\DecValTok{3}\NormalTok{ =}\StringTok{ }\KeywordTok{matrix.power}\NormalTok{(P,}\DecValTok{3}\NormalTok{)}
\KeywordTok{print}\NormalTok{(P_}\DecValTok{3}\NormalTok{)}
\end{Highlighting}
\end{Shaded}

\begin{verbatim}
##       0     1     2
## 0 0.478 0.264 0.258
## 1 0.360 0.256 0.384
## 2 0.570 0.180 0.250
\end{verbatim}

\begin{Shaded}
\begin{Highlighting}[]
\KeywordTok{print}\NormalTok{(P_}\DecValTok{3}\NormalTok{[}\DecValTok{1}\NormalTok{,}\DecValTok{2}\NormalTok{])}
\end{Highlighting}
\end{Shaded}

\begin{verbatim}
## [1] 0.264
\end{verbatim}

slik at: \(P(x_{3}=1|x_{0}=0) = \underline{\underline{0.008}}\)

samme egenskap gjelder også for \(P^{(3)}=P^{3}\)

\begin{Shaded}
\begin{Highlighting}[]
\NormalTok{P_}\DecValTok{4}\NormalTok{ =}\StringTok{ }\KeywordTok{matrix.power}\NormalTok{(P,}\DecValTok{4}\NormalTok{)}
\KeywordTok{print}\NormalTok{(P_}\DecValTok{4}\NormalTok{)}
\end{Highlighting}
\end{Shaded}

\begin{verbatim}
##        0      1      2
## 0 0.4636 0.2540 0.2824
## 1 0.4440 0.2256 0.3304
## 2 0.5240 0.2220 0.2540
\end{verbatim}

\begin{Shaded}
\begin{Highlighting}[]
\KeywordTok{print}\NormalTok{(P_}\DecValTok{4}\NormalTok{[}\DecValTok{1}\NormalTok{,}\DecValTok{2}\NormalTok{])}
\end{Highlighting}
\end{Shaded}

\begin{verbatim}
## [1] 0.254
\end{verbatim}

og vi får da:

\(P(x_{3}=1|x_{0}=0) = \underline{\underline{0.0016}}\)

\begin{enumerate}
\def\labelenumi{\alph{enumi})}
\setcounter{enumi}{1}
\tightlist
\item
  As we se from the graphic presentation below, it is possible to go
  from any state to another trough one or more steps:
\end{enumerate}

\begin{Shaded}
\begin{Highlighting}[]
\KeywordTok{library}\NormalTok{(diagram)}
\end{Highlighting}
\end{Shaded}

\begin{verbatim}
## Loading required package: shape
\end{verbatim}

\begin{Shaded}
\begin{Highlighting}[]
\CommentTok{## Loading required package: shape}
\KeywordTok{plotmat}\NormalTok{(P, }\DataTypeTok{box.size =} \FloatTok{0.028}\NormalTok{, }\DataTypeTok{arr.lcol=}\StringTok{'blue'}\NormalTok{)}
\end{Highlighting}
\end{Shaded}

\includegraphics{STAT220-Oblig-4-5_files/figure-latex/unnamed-chunk-4-1.pdf}

from state 0: \(0 \rightarrow{2} \rightarrow{1}̋\)

from state 1: \(1 \rightarrow{0} \rightarrow{2}\)

from state 2: \(2 \rightarrow{1} \rightarrow{0}\)̋and
\(2 \rightarrow{0}̋\)

It is therfore irreducible.

\end{document}
