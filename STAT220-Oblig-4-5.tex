\PassOptionsToPackage{unicode=true}{hyperref} % options for packages loaded elsewhere
\PassOptionsToPackage{hyphens}{url}
%
\documentclass[]{article}
\usepackage{lmodern}
\usepackage{amssymb,amsmath}
\usepackage{ifxetex,ifluatex}
\usepackage{fixltx2e} % provides \textsubscript
\ifnum 0\ifxetex 1\fi\ifluatex 1\fi=0 % if pdftex
  \usepackage[T1]{fontenc}
  \usepackage[utf8]{inputenc}
  \usepackage{textcomp} % provides euro and other symbols
\else % if luatex or xelatex
  \usepackage{unicode-math}
  \defaultfontfeatures{Ligatures=TeX,Scale=MatchLowercase}
\fi
% use upquote if available, for straight quotes in verbatim environments
\IfFileExists{upquote.sty}{\usepackage{upquote}}{}
% use microtype if available
\IfFileExists{microtype.sty}{%
\usepackage[]{microtype}
\UseMicrotypeSet[protrusion]{basicmath} % disable protrusion for tt fonts
}{}
\IfFileExists{parskip.sty}{%
\usepackage{parskip}
}{% else
\setlength{\parindent}{0pt}
\setlength{\parskip}{6pt plus 2pt minus 1pt}
}
\usepackage{hyperref}
\hypersetup{
            pdftitle={STAT220 Oblig 4-5},
            pdfauthor={Sigbjørn Fjelland},
            pdfborder={0 0 0},
            breaklinks=true}
\urlstyle{same}  % don't use monospace font for urls
\usepackage[margin=1in]{geometry}
\usepackage{color}
\usepackage{fancyvrb}
\newcommand{\VerbBar}{|}
\newcommand{\VERB}{\Verb[commandchars=\\\{\}]}
\DefineVerbatimEnvironment{Highlighting}{Verbatim}{commandchars=\\\{\}}
% Add ',fontsize=\small' for more characters per line
\usepackage{framed}
\definecolor{shadecolor}{RGB}{248,248,248}
\newenvironment{Shaded}{\begin{snugshade}}{\end{snugshade}}
\newcommand{\AlertTok}[1]{\textcolor[rgb]{0.94,0.16,0.16}{#1}}
\newcommand{\AnnotationTok}[1]{\textcolor[rgb]{0.56,0.35,0.01}{\textbf{\textit{#1}}}}
\newcommand{\AttributeTok}[1]{\textcolor[rgb]{0.77,0.63,0.00}{#1}}
\newcommand{\BaseNTok}[1]{\textcolor[rgb]{0.00,0.00,0.81}{#1}}
\newcommand{\BuiltInTok}[1]{#1}
\newcommand{\CharTok}[1]{\textcolor[rgb]{0.31,0.60,0.02}{#1}}
\newcommand{\CommentTok}[1]{\textcolor[rgb]{0.56,0.35,0.01}{\textit{#1}}}
\newcommand{\CommentVarTok}[1]{\textcolor[rgb]{0.56,0.35,0.01}{\textbf{\textit{#1}}}}
\newcommand{\ConstantTok}[1]{\textcolor[rgb]{0.00,0.00,0.00}{#1}}
\newcommand{\ControlFlowTok}[1]{\textcolor[rgb]{0.13,0.29,0.53}{\textbf{#1}}}
\newcommand{\DataTypeTok}[1]{\textcolor[rgb]{0.13,0.29,0.53}{#1}}
\newcommand{\DecValTok}[1]{\textcolor[rgb]{0.00,0.00,0.81}{#1}}
\newcommand{\DocumentationTok}[1]{\textcolor[rgb]{0.56,0.35,0.01}{\textbf{\textit{#1}}}}
\newcommand{\ErrorTok}[1]{\textcolor[rgb]{0.64,0.00,0.00}{\textbf{#1}}}
\newcommand{\ExtensionTok}[1]{#1}
\newcommand{\FloatTok}[1]{\textcolor[rgb]{0.00,0.00,0.81}{#1}}
\newcommand{\FunctionTok}[1]{\textcolor[rgb]{0.00,0.00,0.00}{#1}}
\newcommand{\ImportTok}[1]{#1}
\newcommand{\InformationTok}[1]{\textcolor[rgb]{0.56,0.35,0.01}{\textbf{\textit{#1}}}}
\newcommand{\KeywordTok}[1]{\textcolor[rgb]{0.13,0.29,0.53}{\textbf{#1}}}
\newcommand{\NormalTok}[1]{#1}
\newcommand{\OperatorTok}[1]{\textcolor[rgb]{0.81,0.36,0.00}{\textbf{#1}}}
\newcommand{\OtherTok}[1]{\textcolor[rgb]{0.56,0.35,0.01}{#1}}
\newcommand{\PreprocessorTok}[1]{\textcolor[rgb]{0.56,0.35,0.01}{\textit{#1}}}
\newcommand{\RegionMarkerTok}[1]{#1}
\newcommand{\SpecialCharTok}[1]{\textcolor[rgb]{0.00,0.00,0.00}{#1}}
\newcommand{\SpecialStringTok}[1]{\textcolor[rgb]{0.31,0.60,0.02}{#1}}
\newcommand{\StringTok}[1]{\textcolor[rgb]{0.31,0.60,0.02}{#1}}
\newcommand{\VariableTok}[1]{\textcolor[rgb]{0.00,0.00,0.00}{#1}}
\newcommand{\VerbatimStringTok}[1]{\textcolor[rgb]{0.31,0.60,0.02}{#1}}
\newcommand{\WarningTok}[1]{\textcolor[rgb]{0.56,0.35,0.01}{\textbf{\textit{#1}}}}
\usepackage{graphicx,grffile}
\makeatletter
\def\maxwidth{\ifdim\Gin@nat@width>\linewidth\linewidth\else\Gin@nat@width\fi}
\def\maxheight{\ifdim\Gin@nat@height>\textheight\textheight\else\Gin@nat@height\fi}
\makeatother
% Scale images if necessary, so that they will not overflow the page
% margins by default, and it is still possible to overwrite the defaults
% using explicit options in \includegraphics[width, height, ...]{}
\setkeys{Gin}{width=\maxwidth,height=\maxheight,keepaspectratio}
\setlength{\emergencystretch}{3em}  % prevent overfull lines
\providecommand{\tightlist}{%
  \setlength{\itemsep}{0pt}\setlength{\parskip}{0pt}}
\setcounter{secnumdepth}{0}
% Redefines (sub)paragraphs to behave more like sections
\ifx\paragraph\undefined\else
\let\oldparagraph\paragraph
\renewcommand{\paragraph}[1]{\oldparagraph{#1}\mbox{}}
\fi
\ifx\subparagraph\undefined\else
\let\oldsubparagraph\subparagraph
\renewcommand{\subparagraph}[1]{\oldsubparagraph{#1}\mbox{}}
\fi

% set default figure placement to htbp
\makeatletter
\def\fps@figure{htbp}
\makeatother


\title{STAT220 Oblig 4-5}
\author{Sigbjørn Fjelland}
\date{10/7/2020}

\begin{document}
\maketitle

Library in use:

\begin{Shaded}
\begin{Highlighting}[]
\KeywordTok{library}\NormalTok{(tinytex)}
\KeywordTok{library}\NormalTok{(matrixcalc)}
\KeywordTok{library}\NormalTok{(shape)}
\KeywordTok{library}\NormalTok{(diagram)}
\KeywordTok{library}\NormalTok{(igraph)}
\end{Highlighting}
\end{Shaded}

\begin{verbatim}
## 
## Attaching package: 'igraph'
\end{verbatim}

\begin{verbatim}
## The following object is masked from 'package:matrixcalc':
## 
##     %s%
\end{verbatim}

\begin{verbatim}
## The following objects are masked from 'package:stats':
## 
##     decompose, spectrum
\end{verbatim}

\begin{verbatim}
## The following object is masked from 'package:base':
## 
##     union
\end{verbatim}

\underline{Problem 6.1}

Vi har gitt transition matrix P:

\begin{Shaded}
\begin{Highlighting}[]
\NormalTok{P=}\KeywordTok{matrix}\NormalTok{(}\KeywordTok{c}\NormalTok{(}\FloatTok{0.7}\NormalTok{, }\FloatTok{0.2}\NormalTok{, }\FloatTok{0.1}\NormalTok{, }\FloatTok{0.0}\NormalTok{, }\FloatTok{0.6}\NormalTok{, }\FloatTok{0.4}\NormalTok{, }\FloatTok{0.5}\NormalTok{, }\FloatTok{0.0}\NormalTok{, }\FloatTok{0.5}\NormalTok{), }\DataTypeTok{nrow=}\DecValTok{3}\NormalTok{, }\DataTypeTok{ncol=}\DecValTok{3}\NormalTok{, }\DataTypeTok{byrow =} \OtherTok{TRUE}\NormalTok{)}
\KeywordTok{dimnames}\NormalTok{(P)=}\KeywordTok{list}\NormalTok{(}\KeywordTok{c}\NormalTok{(}\StringTok{'0'}\NormalTok{,}\StringTok{'1'}\NormalTok{,}\StringTok{'2'}\NormalTok{), }\KeywordTok{c}\NormalTok{(}\StringTok{'0'}\NormalTok{,}\StringTok{'1'}\NormalTok{,}\StringTok{'2'}\NormalTok{))}
\KeywordTok{print}\NormalTok{(P)}
\end{Highlighting}
\end{Shaded}

\begin{verbatim}
##     0   1   2
## 0 0.7 0.2 0.1
## 1 0.0 0.6 0.4
## 2 0.5 0.0 0.5
\end{verbatim}

\begin{enumerate}
\def\labelenumi{\alph{enumi})}
\item
  Since \(P_{i,j}^{m} = P(x_{m+n}=j|x_{n})\) and \(P^{(m)}=P^{m}\):

  \(\Rightarrow{P^{(3)}=P^{3}}\) og vi får følgende matrise
\end{enumerate}

\begin{Shaded}
\begin{Highlighting}[]
\NormalTok{P_}\DecValTok{3}\NormalTok{ =}\StringTok{ }\KeywordTok{matrix.power}\NormalTok{(P,}\DecValTok{3}\NormalTok{)}
\KeywordTok{print}\NormalTok{(P_}\DecValTok{3}\NormalTok{)}
\end{Highlighting}
\end{Shaded}

\begin{verbatim}
##       0     1     2
## 0 0.478 0.264 0.258
## 1 0.360 0.256 0.384
## 2 0.570 0.180 0.250
\end{verbatim}

\begin{Shaded}
\begin{Highlighting}[]
\KeywordTok{print}\NormalTok{(P_}\DecValTok{3}\NormalTok{[}\DecValTok{1}\NormalTok{,}\DecValTok{2}\NormalTok{])}
\end{Highlighting}
\end{Shaded}

\begin{verbatim}
## [1] 0.264
\end{verbatim}

slik at: \(P(x_{3}=1|x_{0}=0) = \underline{\underline{0.264}}\)

samme egenskap gjelder også for \(P^{(3)}=P^{3}\)

\begin{Shaded}
\begin{Highlighting}[]
\NormalTok{P_}\DecValTok{4}\NormalTok{ =}\StringTok{ }\KeywordTok{matrix.power}\NormalTok{(P,}\DecValTok{4}\NormalTok{)}
\KeywordTok{print}\NormalTok{(P_}\DecValTok{4}\NormalTok{)}
\end{Highlighting}
\end{Shaded}

\begin{verbatim}
##        0      1      2
## 0 0.4636 0.2540 0.2824
## 1 0.4440 0.2256 0.3304
## 2 0.5240 0.2220 0.2540
\end{verbatim}

\begin{Shaded}
\begin{Highlighting}[]
\KeywordTok{print}\NormalTok{(P_}\DecValTok{4}\NormalTok{[}\DecValTok{1}\NormalTok{,}\DecValTok{2}\NormalTok{])}
\end{Highlighting}
\end{Shaded}

\begin{verbatim}
## [1] 0.254
\end{verbatim}

og vi får da:

\(P(x_{3}=1|x_{0}=0) = \underline{\underline{0.254}}\)

\begin{enumerate}
\def\labelenumi{\alph{enumi})}
\setcounter{enumi}{1}
\tightlist
\item
  As we se from the graphic presentation below, it is possible to go
  from any state to another trough one or more steps:
\end{enumerate}

\begin{Shaded}
\begin{Highlighting}[]
\NormalTok{h <-}\StringTok{ }\KeywordTok{graph_from_adjacency_matrix}\NormalTok{(P, }\DataTypeTok{weighted =} \StringTok{"prob"}\NormalTok{)}
\KeywordTok{E}\NormalTok{(h)}\OperatorTok{$}\NormalTok{prob <-}\StringTok{ }\KeywordTok{ifelse}\NormalTok{(}\KeywordTok{is.nan}\NormalTok{(}\KeywordTok{E}\NormalTok{(h)}\OperatorTok{$}\NormalTok{prob), }\OtherTok{NA}\NormalTok{, }\KeywordTok{E}\NormalTok{(h)}\OperatorTok{$}\NormalTok{prob)}
\KeywordTok{plot}\NormalTok{(h, }\DataTypeTok{edge.label =} \KeywordTok{round}\NormalTok{(}\KeywordTok{E}\NormalTok{(h)}\OperatorTok{$}\NormalTok{prob, }\DecValTok{2}\NormalTok{), }\DataTypeTok{edge.arrow.size =} \FloatTok{.25}\NormalTok{, }\DataTypeTok{edge.curved=}\OperatorTok{-}\FloatTok{0.2}\NormalTok{, }\DataTypeTok{edge.label.cex =} \FloatTok{.5}\NormalTok{)}
\end{Highlighting}
\end{Shaded}

\includegraphics{STAT220-Oblig-4-5_files/figure-latex/unnamed-chunk-5-1.pdf}

from state 0: \(0\rightarrow 2\rightarrow 1\)

from state 1: \(1\rightarrow 0\rightarrow 2\)

from state 2:() \(2\rightarrow 1\rightarrow 0\) and \(1\rightarrow 0\)

It is therfore irreducible.

Periodicity: Since the period of a state is the largest \(d\) that
satisfy following properties: - \(p_{ii}^{(n)} = 0\) whenever \(n\) is
not deviceble by \(d\). - The period \(i\) is shown by \(d(i)\). - If
\(P_{ii}^{(n)} = 0\), for all \(n>0\rightarrow{i}=\infty\)

\begin{verbatim}
  and...
  
\end{verbatim}

\(i\) is periodic if \(d(i)>1\) and aperiodic if \(di=1\)

It is aperiodic since there are several sequences of steps to go from a
state and back again (\(i\rightarrow i\)), icluding the fact that all
the states are self periodic the markov chain is aperiodic.

Transient or recurrent? \(\rightarrow\) It is recurrent! It might stop
and loop at a position or between state 2 and 0 wich are comunicating,
but eventually it will occure in any of the states by a certanty of
100\% as in the formal definition:
\(f_{ii}=P(X_{n} = i, \textrm{for some } n\geq|w_{0}=i)\) for any state
\(i\) is Recurrent if \(f_{ii}=1\) and transient if Recurrent if
\(f_{ii}<1\).

\underline{Problem 6.2}

\begin{enumerate}
\def\labelenumi{\alph{enumi})}
\tightlist
\item
  \(G: Green\) \(R: Red\)
\end{enumerate}

We start initiatly at \(GR\) and pick a random ball with \(p=1/2\). We
ar then sent to a state of \(RR\) or \(GG\) where the probability of
being returned to State \(GR\) is \(100\%\)

\begin{Shaded}
\begin{Highlighting}[]
\NormalTok{A=}\KeywordTok{matrix}\NormalTok{(}\KeywordTok{c}\NormalTok{(}\DecValTok{0}\NormalTok{, }\DecValTok{1}\NormalTok{, }\DecValTok{0}\NormalTok{, }\FloatTok{0.5}\NormalTok{, }\DecValTok{0}\NormalTok{, }\FloatTok{0.5}\NormalTok{, }\DecValTok{0}\NormalTok{, }\DecValTok{1}\NormalTok{, }\DecValTok{0}\NormalTok{), }\DataTypeTok{nrow=}\DecValTok{3}\NormalTok{, }\DataTypeTok{ncol=}\DecValTok{3}\NormalTok{, }\DataTypeTok{byrow =} \OtherTok{TRUE}\NormalTok{)}
\KeywordTok{dimnames}\NormalTok{(A)=}\KeywordTok{list}\NormalTok{(}\KeywordTok{c}\NormalTok{(}\StringTok{'GG'}\NormalTok{,}\StringTok{'GR'}\NormalTok{,}\StringTok{'RR'}\NormalTok{), }\KeywordTok{c}\NormalTok{(}\StringTok{'GG'}\NormalTok{,}\StringTok{'GR'}\NormalTok{,}\StringTok{'RR'}\NormalTok{))}
\end{Highlighting}
\end{Shaded}

\begin{Shaded}
\begin{Highlighting}[]
\NormalTok{f <-}\StringTok{ }\KeywordTok{graph_from_adjacency_matrix}\NormalTok{(A, }\DataTypeTok{weighted =} \StringTok{"prob"}\NormalTok{)}
\KeywordTok{E}\NormalTok{(f)}\OperatorTok{$}\NormalTok{prob <-}\StringTok{ }\KeywordTok{ifelse}\NormalTok{(}\KeywordTok{is.nan}\NormalTok{(}\KeywordTok{E}\NormalTok{(f)}\OperatorTok{$}\NormalTok{prob), }\OtherTok{NA}\NormalTok{, }\KeywordTok{E}\NormalTok{(f)}\OperatorTok{$}\NormalTok{prob)}
\KeywordTok{plot}\NormalTok{(f, }\DataTypeTok{edge.label =} \KeywordTok{round}\NormalTok{(}\KeywordTok{E}\NormalTok{(f)}\OperatorTok{$}\NormalTok{prob, }\DecValTok{2}\NormalTok{),}\DataTypeTok{vertex.size=}\DecValTok{40}\NormalTok{, }\DataTypeTok{edge.arrow.size =} \FloatTok{.25}\NormalTok{, }\DataTypeTok{edge.curved=}\FloatTok{0.3}\NormalTok{, }\DataTypeTok{edge.label.cex =} \FloatTok{.5}\NormalTok{)}
\end{Highlighting}
\end{Shaded}

\includegraphics{STAT220-Oblig-4-5_files/figure-latex/unnamed-chunk-7-1.pdf}

Wich gives following matrix

\begin{Shaded}
\begin{Highlighting}[]
\KeywordTok{print}\NormalTok{(A)}
\end{Highlighting}
\end{Shaded}

\begin{verbatim}
##     GG GR  RR
## GG 0.0  1 0.0
## GR 0.5  0 0.5
## RR 0.0  1 0.0
\end{verbatim}

and following for picking red ball in the future entering the state of
\(GR\rightarrow\) \(P(X_{n}=j|X_{0}=1):\) \#her må jeg få sendt ett
spørsmål

\begin{Shaded}
\begin{Highlighting}[]
\KeywordTok{print}\NormalTok{(A[}\DecValTok{2}\NormalTok{,])}
\end{Highlighting}
\end{Shaded}

\begin{verbatim}
##  GG  GR  RR 
## 0.5 0.0 0.5
\end{verbatim}

\begin{enumerate}
\def\labelenumi{\alph{enumi})}
\setcounter{enumi}{1}
\tightlist
\item
  \(d(GR)=2\) og \(d(GG)=d(RR)=4\) som gir største felles nevner 2 og
  dermed er den periodisk med periode 2.
\end{enumerate}

\underline{Problem 6.3}

Due to previous use of P as variable, it is here substituted for B a)

\begin{Shaded}
\begin{Highlighting}[]
\NormalTok{B=}\KeywordTok{matrix}\NormalTok{(}\KeywordTok{c}\NormalTok{((}\DecValTok{1}\OperatorTok{/}\DecValTok{3}\NormalTok{), }\FloatTok{0.0}\NormalTok{, (}\DecValTok{1}\OperatorTok{/}\DecValTok{3}\NormalTok{), }\FloatTok{0.0}\NormalTok{, }\FloatTok{0.0}\NormalTok{, (}\DecValTok{1}\OperatorTok{/}\DecValTok{3}\NormalTok{), }
\NormalTok{           (}\DecValTok{1}\OperatorTok{/}\DecValTok{2}\NormalTok{), (}\DecValTok{1}\OperatorTok{/}\DecValTok{4}\NormalTok{), (}\DecValTok{1}\OperatorTok{/}\DecValTok{4}\NormalTok{), }\FloatTok{0.0}\NormalTok{, }\FloatTok{0.0}\NormalTok{, }\FloatTok{0.0}\NormalTok{, }
           \FloatTok{0.0}\NormalTok{, }\FloatTok{0.0}\NormalTok{, (}\DecValTok{1}\OperatorTok{/}\DecValTok{2}\NormalTok{), }\FloatTok{0.0}\NormalTok{, (}\DecValTok{1}\OperatorTok{/}\DecValTok{2}\NormalTok{), }\FloatTok{0.0}\NormalTok{,}
\NormalTok{           (}\DecValTok{1}\OperatorTok{/}\DecValTok{4}\NormalTok{), (}\DecValTok{1}\OperatorTok{/}\DecValTok{4}\NormalTok{), (}\DecValTok{1}\OperatorTok{/}\DecValTok{4}\NormalTok{), }\FloatTok{0.0}\NormalTok{, }\FloatTok{0.0}\NormalTok{, (}\DecValTok{1}\OperatorTok{/}\DecValTok{4}\NormalTok{),}
           \FloatTok{0.0}\NormalTok{, }\FloatTok{0.0}\NormalTok{, }\FloatTok{1.0}\NormalTok{, }\FloatTok{0.0}\NormalTok{, }\FloatTok{0.0}\NormalTok{, }\FloatTok{0.0}\NormalTok{, }
           \FloatTok{0.0}\NormalTok{, }\FloatTok{0.0}\NormalTok{, }\FloatTok{0.0}\NormalTok{, }\FloatTok{0.0}\NormalTok{, }\FloatTok{0.0}\NormalTok{, }\FloatTok{1.0}\NormalTok{), }\DataTypeTok{nrow=}\DecValTok{6}\NormalTok{, }\DataTypeTok{ncol=}\DecValTok{6}\NormalTok{, }\DataTypeTok{byrow =} \OtherTok{TRUE}\NormalTok{)}
\KeywordTok{dimnames}\NormalTok{(B)=}\KeywordTok{list}\NormalTok{(}\KeywordTok{c}\NormalTok{(}\StringTok{'0'}\NormalTok{,}\StringTok{'1'}\NormalTok{,}\StringTok{'2'}\NormalTok{,}\StringTok{'3'}\NormalTok{,}\StringTok{'4'}\NormalTok{,}\StringTok{'5'}\NormalTok{), }\KeywordTok{c}\NormalTok{(}\StringTok{'0'}\NormalTok{,}\StringTok{'1'}\NormalTok{,}\StringTok{'2'}\NormalTok{,}\StringTok{'3'}\NormalTok{,}\StringTok{'4'}\NormalTok{,}\StringTok{'5'}\NormalTok{))}
\end{Highlighting}
\end{Shaded}

\begin{Shaded}
\begin{Highlighting}[]
\NormalTok{g <-}\StringTok{ }\KeywordTok{graph_from_adjacency_matrix}\NormalTok{(B, }\DataTypeTok{weighted =} \StringTok{"prob"}\NormalTok{)}
\KeywordTok{E}\NormalTok{(g)}\OperatorTok{$}\NormalTok{prob <-}\StringTok{ }\KeywordTok{ifelse}\NormalTok{(}\KeywordTok{is.nan}\NormalTok{(}\KeywordTok{E}\NormalTok{(g)}\OperatorTok{$}\NormalTok{prob), }\OtherTok{NA}\NormalTok{, }\KeywordTok{E}\NormalTok{(g)}\OperatorTok{$}\NormalTok{prob)}
\KeywordTok{plot}\NormalTok{(g, }\DataTypeTok{edge.label =} \KeywordTok{round}\NormalTok{(}\KeywordTok{E}\NormalTok{(g)}\OperatorTok{$}\NormalTok{prob, }\DecValTok{2}\NormalTok{), }\DataTypeTok{edge.arrow.size =} \FloatTok{.25}\NormalTok{, }\DataTypeTok{edge.label.cex =} \FloatTok{.5}\NormalTok{)}
\end{Highlighting}
\end{Shaded}

\includegraphics{STAT220-Oblig-4-5_files/figure-latex/unnamed-chunk-11-1.pdf}

\newpage

\begin{enumerate}
\def\labelenumi{\alph{enumi})}
\setcounter{enumi}{1}
\item
  \underline{As we se from the graph above:}

  We have three recurrent states, 2, 4 and 5, devided in two classes.\\
  class 1 - State 5 is self-recurrent class 2 - State 2 and 4
\end{enumerate}

\textbf{Transient States:}

State 0, 1, and 3 will at some pint possibly loop, but ultimatly it will
end up in recurrent state class 1 or 2.

\begin{enumerate}
\def\labelenumi{\alph{enumi})}
\setcounter{enumi}{2}
\item
  \underline{Communicating States:}

  \((2\leftrightarrow{4})\), \((1\leftrightarrow{1})\),
  \((2\leftrightarrow{2})\) and \((5\leftrightarrow{5})\)

  \begin{enumerate}
  \def\labelenumii{\alph{enumii})}
  \setcounter{enumii}{3}
  \tightlist
  \item
    with statespace = \{1,3,0,2,4,5\}, we get following probability
    matrix (Denoted - C in code):
  \end{enumerate}
\end{enumerate}

\begin{Shaded}
\begin{Highlighting}[]
\NormalTok{C=}\KeywordTok{matrix}\NormalTok{(}\KeywordTok{c}\NormalTok{((}\DecValTok{1}\OperatorTok{/}\DecValTok{4}\NormalTok{), }\FloatTok{0.0}\NormalTok{, (}\DecValTok{1}\OperatorTok{/}\DecValTok{2}\NormalTok{), (}\DecValTok{1}\OperatorTok{/}\DecValTok{4}\NormalTok{), }\FloatTok{0.0}\NormalTok{, }\FloatTok{0.0}\NormalTok{, }
           \FloatTok{0.0}\NormalTok{, }\FloatTok{0.0}\NormalTok{, }\FloatTok{0.0}\NormalTok{, (}\DecValTok{1}\OperatorTok{/}\DecValTok{2}\NormalTok{), (}\DecValTok{1}\OperatorTok{/}\DecValTok{2}\NormalTok{), }\FloatTok{0.0}\NormalTok{, }
           \FloatTok{0.0}\NormalTok{, }\FloatTok{0.0}\NormalTok{, (}\DecValTok{1}\OperatorTok{/}\DecValTok{3}\NormalTok{), (}\DecValTok{1}\OperatorTok{/}\DecValTok{3}\NormalTok{), }\FloatTok{0.0}\NormalTok{, (}\DecValTok{1}\OperatorTok{/}\DecValTok{3}\NormalTok{),}
           \FloatTok{0.0}\NormalTok{, }\FloatTok{0.0}\NormalTok{, }\FloatTok{0.0}\NormalTok{, (}\DecValTok{1}\OperatorTok{/}\DecValTok{2}\NormalTok{), (}\DecValTok{1}\OperatorTok{/}\DecValTok{2}\NormalTok{), }\FloatTok{0.0}\NormalTok{,}
           \FloatTok{0.0}\NormalTok{, }\FloatTok{0.0}\NormalTok{, }\FloatTok{0.0}\NormalTok{, }\FloatTok{1.0}\NormalTok{, }\FloatTok{0.0}\NormalTok{, }\FloatTok{0.0}\NormalTok{, }
           \FloatTok{0.0}\NormalTok{, }\FloatTok{0.0}\NormalTok{, }\FloatTok{0.0}\NormalTok{, }\FloatTok{0.0}\NormalTok{, }\FloatTok{0.0}\NormalTok{, }\FloatTok{1.0}\NormalTok{), }\DataTypeTok{nrow=}\DecValTok{6}\NormalTok{, }\DataTypeTok{ncol=}\DecValTok{6}\NormalTok{, }\DataTypeTok{byrow =} \OtherTok{TRUE}\NormalTok{)}
\KeywordTok{dimnames}\NormalTok{(C)=}\KeywordTok{list}\NormalTok{(}\KeywordTok{c}\NormalTok{(}\StringTok{'1'}\NormalTok{,}\StringTok{'3'}\NormalTok{,}\StringTok{'0'}\NormalTok{,}\StringTok{'2'}\NormalTok{,}\StringTok{'4'}\NormalTok{,}\StringTok{'5'}\NormalTok{), }\KeywordTok{c}\NormalTok{(}\StringTok{'1'}\NormalTok{,}\StringTok{'3'}\NormalTok{,}\StringTok{'0'}\NormalTok{,}\StringTok{'2'}\NormalTok{,}\StringTok{'4'}\NormalTok{,}\StringTok{'5'}\NormalTok{))}

\KeywordTok{print}\NormalTok{(C)}
\end{Highlighting}
\end{Shaded}

\begin{verbatim}
##      1 3         0         2   4         5
## 1 0.25 0 0.5000000 0.2500000 0.0 0.0000000
## 3 0.00 0 0.0000000 0.5000000 0.5 0.0000000
## 0 0.00 0 0.3333333 0.3333333 0.0 0.3333333
## 2 0.00 0 0.0000000 0.5000000 0.5 0.0000000
## 4 0.00 0 0.0000000 1.0000000 0.0 0.0000000
## 5 0.00 0 0.0000000 0.0000000 0.0 1.0000000
\end{verbatim}

In a nicer way in Latex: ``\[ A = " %_% C %_% "\]''

\begin{enumerate}
\def\labelenumi{\alph{enumi})}
\setcounter{enumi}{4}
\tightlist
\item
  The only recurrent state is State 5, therefore the probability of
  hitting recurrent state is given by vector:
\end{enumerate}

\begin{Shaded}
\begin{Highlighting}[]
\KeywordTok{print}\NormalTok{(B[,}\DecValTok{6}\NormalTok{])}
\end{Highlighting}
\end{Shaded}

\begin{verbatim}
##         0         1         2         3         4         5 
## 0.3333333 0.0000000 0.0000000 0.2500000 0.0000000 1.0000000
\end{verbatim}

This gives \(P_{0}(X_{T}=5)=P(X_{T}=5|X_{T-1}=0)=0.3333=\frac{1}{3}\)
and the expectation \(E(X_{T}=5|X_{T-1}=0)\)

\end{document}
